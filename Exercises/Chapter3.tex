\documentclass[12pt, a4paper]{article}

\usepackage{amsmath}
\usepackage{amssymb}
\usepackage{titlesec}
\usepackage{tcolorbox}
\usepackage{enumitem}
\usepackage[bottom=3cm]{geometry}
\usepackage[hidelinks]{hyperref}
\tcbuselibrary{breakable}
\tcbuselibrary{skins}

\newtcolorbox{prob}[1]{colback=gray!5!white, colframe=gray!75!black, 
title=\textbf{Exercise #1}}

\newtcolorbox{sol}{
    breakable,
    colback=white,      % Background matches page
    colframe=white,     % Frame matches page (invisible)
    frame hidden,       % Hides the border line
    left=3mm, right=3mm,% MATCHES the Question box padding
    boxrule=0mm,        % No border width
    top=0mm, bottom=0mm,% Tight vertical spacing
    parbox=false,       % Allows paragraphs to break normally
    before upper={\textbf{Solution:}\par\medskip} % Automatically adds "Solution:" title
}

\begin{document}

\begin{prob}{3.1}
    A die is selected at random from two twenty-faced dice on which the symbols 1--10 are written with nonuniform frequency as follows.

\begin{center}
\begin{tabular}{lcccccccccc}
\hline
Symbol & 1 & 2 & 3 & 4 & 5 & 6 & 7 & 8 & 9 & 10 \\
\hline
Number of faces of die A & 6 & 4 & 3 & 2 & 1 & 1 & 1 & 1 & 1 & 0 \\
Number of faces of die B & 3 & 3 & 2 & 2 & 2 & 2 & 2 & 2 & 1 & 1 \\
\hline
\end{tabular}
\end{center}

The randomly chosen die is rolled 7 times, with the following outcomes:
\[ 5, 3, 9, 3, 8, 4, 7. \]
What is the probability that the die is die A?
\end{prob}
\begin{sol}
    \begin{align*}
        P(S|A) &= \frac{1}{20} \times \frac{3}{20} \times \frac{1}{20} \times \frac{3}{20} \times \frac{1}{20} \times \frac{2}{20} \times  \frac{1}{20} \\
        &= \frac{18}{20^7} \\
        P(S|B) &= \frac{2}{20} \times \frac{2}{20} \times \frac{1}{20} \times \frac{2}{20} \times \frac{2}{20} \times \frac{2}{20} \times \frac{2}{20} \\
        &= \frac{64}{20^7} \\
        P(A|S) &= \frac{\frac{18}{20^7} \times \frac{1}{2}}{\frac{18}{20^7} \times \frac{1}{2} + \frac{64}{20^7} \times \frac{1}{2}} \\
        &= \frac{9}{41} \\
    \end{align*}
\end{sol}
\bigskip


\begin{prob}{3.2}
    Assume that there is a third twenty-faced die, die C, on which the symbols 1--20 are written once each. As above, one of the three dice is selected at random and rolled 7 times, giving the outcomes:
\[ 3, 5, 4, 8, 3, 9, 7. \]
What is the probability that the die is (a) die A, (b) die B, (c) die C?
\end{prob}
\begin{sol}
    \begin{align*}
        P(S|A) &= \frac{3 \times 1 \times 2 \times 1 \times 3 \times 1 \times 1}{20^7} = \frac{18}{20^7} \\
        P(S|B) &= \frac{2 \times 2 \times 2 \times 2 \times 2 \times 1 \times 2}{20^7} = \frac{64}{20^7} \\
        P(S|C) &= \frac{1}{20^7}
    \end{align*}
    \begin{enumerate}
        \item[(i)] $$P(A|S) = \frac{18}{18+64+1} = \frac{18}{83}$$ 
        \item[(ii)]$$P(B|S) = \frac{64}{18+64+1} = \frac{64}{83}$$ 
        \item[(iii)]  $$P(C|S) = \frac{1}{18+64+1} = \frac{1}{83}$$ 
    \end{enumerate}
\end{sol}
\bigskip

\begin{prob}{3.3}
    \textbf{Inferring a decay constant} \\
Unstable particles are emitted from a source and decay at a distance $x$, a real number that has an exponential probability distribution with characteristic length $\lambda$. Decay events can be observed only if they occur in a window extending from $x = 1\,\text{cm}$ to $x = 20\,\text{cm}$. $N$ decays are observed at locations $\{x_1, \dots, x_N\}$. What is $\lambda$?
\end{prob}
\begin{sol}
    \begin{align*}
        P(\lambda | \{ x_1, \dots , x_N\}) &= \frac{P(\{ x_1, \dots , x_N\} | \lambda) P(\lambda)}{P(\{ x_1, \dots , x_N\})} \\
        &= \frac{P(x_1|\lambda)P(x_2|\lambda)\dots P(x_N|\lambda)P(\lambda)}{P(\{ x_1, \dots , x_N\})}
    \end{align*}
    Now, 
\begin{align*}
    P(x|\lambda)
\begin{cases} 
  \frac{1}{\lambda} e^{-\frac{x}{\lambda}} & \text{if } 1 \leq x \leq 20 \\
  0 & \text{otherwise}
\end{cases}
\end{align*}
and if we let 
\begin{align*}
    Z(\lambda) &- \int_{1}^{20} \frac{1}{\lambda} e^{-\frac{x}{\lambda}} dx \\
    &= \left[-e^{-\frac{x}{\lambda}} \right]_1^20 \\
    &= e^{-\frac{1}{\lambda}} - e^{-\frac{20}{\lambda}} \\
\end{align*}
then
\begin{align*}
    P(\lambda | \{ x_1, \dots , x_N\}) &= \frac{\frac{1}{\lambda} e ^{-\frac{x_1}{\lambda}} \times \frac{1}{\lambda} e ^{-\frac{x_2}{\lambda}} \times \dots \times \frac{1}{\lambda} e ^{-\frac{x_N}{\lambda}}}{(Z(\lambda))^N} P(\lambda) \\
    &= \frac{e^{-\sum_{1}^{N} \frac{x_N}{\lambda}}P(\lambda)}{(\lambda Z(\lambda))^N}
\end{align*}
\end{sol}
\bigskip

\begin{prob}{3.4}
    \textbf{Forensic evidence} \\
Two people have left traces of their own blood at the scene of a crime. A suspect, Oliver, is tested and found to have type `O' blood. The blood groups of the two traces are found to be of type `O' (a common type in the local population, having frequency 60\%) and of type `AB' (a rare type, with frequency 1\%). Do these data (type `O' and `AB' blood were found at scene) give evidence in favour of the proposition that Oliver was one of the two people present at the crime?
\end{prob}
\begin{sol}
    Let $O$ be the case that Oliver is the criminal, and $e$ the evidence.
    \begin{align*}
        P(O|e) &= \frac{P(e|O)P(O)}{P(e)} \\
        &= \frac{0.01P(o)}{2 \times 0.01 \times 0.6} \\
        &= \frac{P(O)}{1.2}
    \end{align*}
    so no, the posterior is lower than the prior, so the evidence is not in favour of the proposition that Oliver was one of the two people present at the crime.
\end{sol}
\bigskip

\begin{prob}{3.5}
    Sketch the posterior probability Sketch the posterior probability $P(p_{\text{a}} \mid \mathbf{s} = \texttt{aba}, F = 3)$. \\
What is the most probable value of $p_{\text{a}}$ (i.e., the value that maximizes the posterior probability density)? What is the mean value of $p_{\text{a}}$ under this distribution?

\end{prob}
\begin{sol}
    Using (3.11) and (3.12) from the textbook, 
    \begin{align*}
        P(p_{\text{a}} \mid \mathbf{s} = \texttt{aba}, F = 3) &= \frac{p_a^2 (1-p_a)}{\frac{2!}{4!}} = 12p_a^2 (1-p_a) \\
        P(p_{\text{a}} \mid \mathbf{s} = \texttt{bbb}, F = 3) &= \frac{(1-p_a)^3}{\frac{3!}{4!}} = 4(1-p_a)^3 \\
    \end{align*}
    The graph is available in the following link: \href{https://www.desmos.com/calculator/falltenjzo}{www.desmos.com/calculator/falltenjzo}
\end{sol}
\bigskip

\begin{prob}{3.6}
    Show that after $F$ tosses have taken place, the biggest value that the log evidence ratio \\
\begin{equation}
    \log \frac{P(\mathbf{s} | F, \mathcal{H}_1)}{P(\mathbf{s} | F, \mathcal{H}_0)} \tag{3.23}
\end{equation}
can have scales \textit{linearly} with $F$ if $\mathcal{H}_1$ is more probable, but the log evidence in favour of $\mathcal{H}_0$ can grow at most as $\log F$.
\end{prob}
\begin{sol}
    From (3.12) and (3.20) in textbook,
\begin{align*}
    & \log \frac{P(\mathbf{s} | F, \mathcal{H}_1)}{P(\mathbf{s} | F, \mathcal{H}_0)} \\
    =& \log \frac{F_a!F_b!}{(F_a+F_b+1)!} - \log p_0^{F_a}(1-p_0)^{F_b} \\
    =& \log F_a! + \log F_b! - \log (F_a + F_b + 1)! - F_a \log p_0 - F_b \log (1-p_0) \\
    =& \log F_a! + \log F_b! - \log (F+1) - \log F! - F_a \log p_0 - F_b \log (1-p_0) \quad (\because F = F_a + F_b) \\
    \approx& F_a \log F_a - F_a + \frac{1}{2} \log(2\pi F_a) + F_b \log F_b - F_b + \frac{1}{2} \log(2\pi F_b) - \log F \\
    & \quad - F \log F + F - \frac{1}{2} \log (2\pi F) \quad (\because \text{Stirling's approximation and } F \approx F + 1 \text{ for large F})\\
    &= F_a \log \frac{F_a}{p_0} + F_b \log \frac{F_b}{1-p_0} - F \log F + \frac{1}{2} \log (2\pi \frac{F_a F_b}{F}) \\
\end{align*}
Now, let $f_a = \frac{F_a}{F}$, $f_b = \frac{F_b}{F}$ then
\begin{align*}
    &F_a \log \frac{F_a}{p_0} + F_b \log \frac{F_b}{1-p_0} - F \log F - \log F + \frac{1}{2} \log (2\pi \frac{F_a F_b}{F})  
    \\=& F \left( f_a \log \frac{f_a}{p_0} + f_b \log \frac{f_b}{1-p_0} \right) - \frac{1}{2} \log F + \frac{1}{2} \log (2\pi f_a f_b)
\end{align*}

When $\mathcal{H}_1$ is more probable, $f_a \neq p_0$ so $f_a \log \frac{f_a}{p_0} + f_b \log \frac{f_b}{1-p_0}$ becomes a constant, hence log evidence ratio scales linearly with $F$. \\
When $\mathcal{H}_0$ is more probable, $f_a \rightarrow p_0$ so $f_a \log \frac{f_a}{p_0} + f_b \log \frac{f_b}{1-p_0}$ becomes 0. $f_a, f_b$ are constant and can be ignored so the second term will dominate. The log evidence ratio grows at most as $\log F$ (negatively). \\

Note $f_a \log \frac{f_a}{p_0} + f_b \log \frac{f_b}{1-p_0} = KL(Q||P)$ where $Q$ is the observed distribution and $P$ is the null hypothesis distribution!

\end{sol}
\bigskip


% template
\begin{prob}{}
\end{prob}
\begin{sol}
\end{sol}
\bigskip


\end{document}